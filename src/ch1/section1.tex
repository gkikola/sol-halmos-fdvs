\chapter{Spaces}

\section{Fields}

\Exercise1 Almost all the laws of elementary arithmetic are
consequences of the axioms defining a field. Prove, in particular,
that if $\F$ is a field, and if $\alpha$, $\beta$, and $\gamma$ belong
to $\F$, then the following relations hold.
\label{exercise:field-properties}
\begin{enumerate}
\item $0 + \alpha = \alpha$.
  \begin{proof}
    By the commutativity of addition and the definition of $0$,
    \begin{equation*}
      0 + \alpha = \alpha + 0 = \alpha. \qedhere
    \end{equation*}
  \end{proof}
\item If $\alpha + \beta = \alpha + \gamma$, then $\beta = \gamma$.
  \begin{proof}
    Adding $-\alpha$ to both sides of the first equation gives
    \begin{equation*}
      (\alpha + \beta) + (-\alpha) = (\alpha + \gamma) + (-\alpha),
    \end{equation*}
    which by associativity and commutativity of addition gives
    \begin{equation*}
      (\alpha + (-\alpha)) + \beta = (\alpha + (-\alpha)) + \gamma,
    \end{equation*}
    and by definition of additive inverses, this gives
    \begin{equation*}
      0 + \beta = 0 + \gamma
      \quad\text{or}\quad
      \beta = \gamma,
    \end{equation*}
    making use of the fact that $0 + \delta = \delta$ for any
    $\delta\in\F$ (already proven above).
  \end{proof}
\item $\alpha + (\beta - \alpha) = \beta$. (Here
  $\beta - \alpha = \beta + (-\alpha)$.)
  \begin{proof}
    This follows from commutativity and associativity of addition:
    \begin{equation*}
      \alpha + (\beta - \alpha)
      = \alpha + (-\alpha + \beta)
      = (\alpha - \alpha) + \beta
      = 0 + \beta
      = \beta. \qedhere
    \end{equation*}
  \end{proof}
\item $\alpha\cdot0 = 0\cdot\alpha = 0$.
  \begin{proof}
    By definition of $0$ and $1$ and by distributivity we have
    \begin{equation*}
      \alpha\cdot0 = \alpha(1 - 1)
      = \alpha\cdot1 - \alpha\cdot1
      = \alpha - \alpha = 0.
    \end{equation*}
    By commutativity of multiplication, $0\cdot\alpha = 0$ as well.
  \end{proof}
\item $(-1)\alpha = -\alpha$.
  \begin{proof}
    From the various field axioms we have
    \begin{align*}
      (-1)\alpha
      &= 0 + (-1)\alpha
        = (\alpha - \alpha) + (-1)\alpha \\
      &= (-\alpha + \alpha) + \alpha(-1) \\
      &= -\alpha + (\alpha\cdot1 + \alpha(-1)) \\
      &= -\alpha + \alpha(1 - 1) \\
      &= -\alpha + \alpha\cdot0 \\
      &= -\alpha + 0
        = -\alpha. \qedhere
    \end{align*}
  \end{proof}
\item $(-\alpha)(-\beta) = \alpha\beta$.
  \begin{proof}
    Since
    \begin{align*}
      (-1)(-1) + (-1) &= (-1)(-1) + (-1)1 \\
                      &= (-1)(-1 + 1) = -1\cdot0 = 0,
    \end{align*}
    it follows that $(-1)(-1)$ is an additive inverse of $-1$. Since
    additive inverses are unique, we have $(-1)(-1) = 1$. Using this
    fact along with the previous result and with commutativity and
    associativity of multiplication we have
    \begin{equation*}
      (-\alpha)(-\beta)
      = ((-1)\alpha)((-1)\beta)
      = ((-1)(-1))(\alpha\beta)
      = 1(\alpha\beta) = \alpha\beta
    \end{equation*}
    as desired.
  \end{proof}
\item If $\alpha\beta = 0$, then either $\alpha = 0$ or $\beta = 0$
  (or both).
  \begin{proof}
    Let $\alpha\beta = 0$. If $\alpha = 0$ then we are done, so
    suppose $\alpha$ is nonzero. Then $\alpha$ has a unique
    multiplicative inverse $\alpha^{-1}$. Multiplying both sides of
    the original equation by this inverse gives
    \begin{equation*}
      \alpha^{-1}(\alpha\beta) = \alpha^{-1}\cdot0
    \end{equation*}
    which gives
    \begin{equation*}
      (\alpha^{-1}\alpha)\beta = 0.
    \end{equation*}
    And since $\alpha^{-1}\alpha = \alpha\alpha^{-1} = 1$ we have
    $\beta = 0$ which completes the proof.
  \end{proof}
\end{enumerate}

\Exercise2
\begin{enumerate}
\item Is the set of all positive integers a field?
  \begin{solution}
    The set of positive integers (note that Halmos defines this set as
    including $0$) is not a field because, for example, $1$ does not
    have an additive inverse in this set.
  \end{solution}
\item What about the set of all integers?
  \begin{solution}
    The set of all integers is not a field since, for example, $2$
    does not have a multiplicative inverse in the set.
  \end{solution}
\item Can the answers to these questions be changed by re-defining
  addition or multiplication (or both)?
  \begin{solution}
    Yes, though the operations can become rather complicated. For
    example, we can form a bijection (a one-to-one correspondence) $f$
    between the integers and the rationals since both are countable
    sets. Then define addition of integers $\oplus$ and multiplication
    of integers $\otimes$ by
    \begin{equation*}
      \alpha\oplus\beta = f^{-1}(f(\alpha)+f(\beta))
    \end{equation*}
    and
    \begin{equation*}
      \alpha\otimes\beta = f^{-1}(f(\alpha)\cdot f(\beta)),
    \end{equation*}
    where $+$ and $\cdot$ indicate the usual operations on the
    rationals. Since the rationals form a field, it is not difficult
    to show that the binary operations $\oplus$ and $\otimes$ make the
    integers into a field with $f^{-1}(0)$ taking the role of the
    additive identity and $f^{-1}(1)$ taking the role of the
    multiplicative identity.
  \end{solution}
\end{enumerate}

\Exercise3 Let $m$ be an integer, $m\geq2$, and let $\Z_m$ be the set
of all positive integers less than $m$,
\begin{equation*}
  \Z_m = \{0,1,\dots,m-1\}.
\end{equation*}
If $\alpha$ and $\beta$ are in $\Z_m$, let $\alpha + \beta$ be the
least positive remainder obtained by dividing the (ordinary) sum of
$\alpha$ and $\beta$ by $m$, and, similarly, let $\alpha\beta$ be the
least positive remainder obtained by dividing the (ordinary) product
of $\alpha$ and $\beta$ by $m$.
\begin{enumerate}
\item Prove that $\Z_m$ is a field if and only if $m$ is a prime.
  \begin{proof}
    Note that addition and multiplication, as defined here, are both
    closed since dividing by $m$ will always produce a remainder
    between $0$ and $m - 1$. Note also that commutativity,
    associativity, and distributivity of these operations follow from
    the respective properties of ordinary addition and multiplication
    (for example, dividing $\alpha + \beta$ by $m$ produces the same
    remainder as dividing $\beta + \alpha$ by $m$).

    Also note that $\Z_m$ contains the additive identity $0$ and the
    multiplicative identity $1$, since $\alpha + 0$, when divided by
    $m$, always produces the remainder $\alpha$ and similarly for
    $1\alpha$. We also have additive inverses since $-\alpha$ divided
    by $m$ produces a remainder of $m - \alpha$, so that
    $\alpha + -\alpha = \alpha + (m - \alpha)$ gives the expected
    remainder $0$.

    Therefore, to show that $\Z_m$ is a field, we only need show that
    every nonzero element has a multiplicative inverse.

    We will make use of some results from number theory. Suppose $m$
    is prime. Then for any nonzero $\alpha\in\Z_m$, the greatest
    common divisor of $\alpha$ and $m$ must be $1$. By B\'ezout's
    Identity, there exist integers $x$ and $y$ such that
    \begin{equation*}
      \alpha x + my = 1,
    \end{equation*}
    (we are here using ordinary addition and multiplication). Then
    $\alpha x = -my + 1$, and it follows that $\alpha x$, when divided
    by $m$, leaves a remainder of $1$. Therefore we can take
    $\alpha^{-1}$ to be the least positive remainder of dividing $x$
    by $m$.

    Finally, to show the converse, note that if $m = ab$ where
    $a,b>1$, then $a,b\in\Z_m$ but $ab = 0$. By an earlier result
    (Exercise~\ref{exercise:field-properties}), if $Z_m$ is a field,
    then $ab = 0$ implies that $a = 0$ or $b = 0$, which is a
    contradiction. Therefore $Z_m$ is not a field in this case.
  \end{proof}
\item What is $-1$ in $\Z_5$?
  \begin{solution}
    The additive inverse of $1$ in $\Z_5$ is $4$, since $1 + 4 = 0$.
  \end{solution}
\item What is $\frac13$ in $\Z_7$?
  \begin{solution}
    The multiplicative inverse of $3$ in $\Z_7$ is $5$ since
    $3\cdot5 = 1$. Therefore $1\cdot3^{-1} = 1\cdot5 = 5$.
  \end{solution}
\end{enumerate}

\Exercise4 The example of $\Z_p$ (where $p$ is prime) shows that not
quite all the laws of elementary arithmetic hold in fields; in $\Z_2$,
for instance, $1 + 1 = 0$. Prove that if $\F$ is a field, then either
the result of repeatedly adding $1$ to itself is always different from
$0$, or else the first time that it is equal to $0$ occurs when the
number of summands is a prime. (The {\em characteristic} of the field
$\F$ is defined to be $0$ in the first case and the crucial prime in
the second.)
\begin{proof}
  For this exercise, let $n\cdot\alpha$ represent the result of adding
  $\alpha$ to itself $n$ times, where $n$ is an ordinary strictly
  positive integer and $\alpha$ is in the field $\F$. If $n\cdot1$ is
  never $0$ for any $n$ then we are done, so suppose there is some
  particular $n$ such that $n\cdot1 = 0$. Obviously $n>1$ since the
  additive and multiplicative identities in a field are distinct by
  definition.

  Suppose $n = ab$, so that $(ab)\cdot1 = 0$. But
  $a\cdot(b\cdot1) = 0$ also, since adding $1$ to itself $b$ times,
  taking the result, and adding it to itself $a$ times is the same as
  just adding $1$ to itself $ab$ times.

  Let $c = b\cdot1$. By distributivity, we can see that
  \begin{equation*}
    a\cdot c
    = \overbrace{c + c + c + \cdots + c}^{\text{$a$ terms}}
    = c(\overbrace{1 + 1 + 1 + \cdots + 1}^{\text{$a$ terms}})
    = c(a\cdot 1).
  \end{equation*}
  Therefore $(b\cdot1)(a\cdot1) = 0$, and since $b\cdot1$ and
  $a\cdot1$ are both in $\F$, we see that either $b\cdot1 = 0$ or
  $a\cdot1 = 0$.

  Now, find the prime factorization of $n$ so that
  \begin{equation*}
    n = p_1^{e_1}p_2^{e_2}\cdots p_k^{e_k},
    \quad\text{where $p_i$ is prime and $e_i\geq1$ for each $i$}.
  \end{equation*}
  From the above argument, we know that either $p_1\cdot1 = 0$, or
  $p_2\cdot1 = 0$, $\dots$, or $p_k\cdot1 = 0$. And each $p_i$ is
  smaller than $n$ (unless $n$ is itself prime), so this shows that no
  matter what value of $n$ we choose such that $n\cdot1 = 0$, we can
  always find a smaller prime $p$ so that $p\cdot1 = 0$. Therefore the
  smallest possible $n$ must be prime, which completes the proof.
\end{proof}

\Exercise5 Let $\Q(\sqrt2)$ be the set of all real numbers of the form
$\alpha + \beta\sqrt2$, where $\alpha$ and $\beta$ are rational.
\begin{enumerate}
\item Is $\Q(\sqrt2)$ a field?
  \begin{solution}
    Since
    \begin{equation*}
      (\alpha_1 + \beta_1\sqrt2) + (\alpha_2 + \beta_2\sqrt2)
      = (\alpha_1 + \alpha_2) + (\beta_1 + \beta_2)\sqrt2
    \end{equation*}
    this shows that $a + b\in\Q(\sqrt2)$ whenever $a$ and $b$ are
    themselves members. Similarly
    \begin{equation*}
      (\alpha_1 + \beta_1\sqrt2)(\alpha_2 + \beta_2\sqrt2)
      = (\alpha_1\alpha_2 + 2\beta_1\beta_2)
      + (\alpha_1\beta_2 + \alpha_2\beta_1)\sqrt2,
    \end{equation*}
    so multiplication is also closed.

    Multiplicative inverses exist since if $\alpha + \beta\sqrt2$ is
    nonzero, then
    \begin{equation*}
      (\alpha + \beta\sqrt2)
      \left(\frac{\alpha}{\alpha^2 - 2\beta^2}
        + \frac{-\beta}{\alpha^2 - 2\beta^2}\sqrt2\right)
      = \frac{(\alpha+\beta\sqrt2)(\alpha-\beta\sqrt2)}
      {\alpha^2 - 2\beta^2} = 1.
    \end{equation*}

    The remaining properties follow from the properties of the
    rationals, with $0 = 0 + 0\sqrt2$ and $1 = 1 + 0\sqrt2$ taking
    their usual roles. Therefore $\Q(\sqrt2)$ is indeed a field.
  \end{solution}
\item What if $\alpha$ and $\beta$ are required to be integers?
  \begin{solution}
    If $\alpha$ and $\beta$ must be integers, then the resulting set
    does not form a field since $2 = 2 + 0\sqrt2$ (for example) does
    not have a multiplicative inverse.
  \end{solution}
\end{enumerate}

\Exercise6
\begin{enumerate}
\item Does the set of all polynomials with integer coefficients form a
  field?
  \begin{solution}
    If the set (call it $\Z[x]$) did form a field, $1$ would have to
    be the multiplicative identity. But there is no polynomial which,
    when multiplied by the polynomial $x$, gives $1$ (that is, $1/x$
    is not a polynomial). Since there is a nonzero element in $\Z[x]$
    which does not have a multiplicative inverse, $\Z[x]$ cannot be a
    field.
  \end{solution}
\item What if the coefficients are allowed to be real numbers?
  \begin{solution}
    This set is still not a field for the same reason.
  \end{solution}
\end{enumerate}

\Exercise7 Let $\F$ be the set of all (ordered) pairs
$(\alpha, \beta)$ of real numbers.
\begin{enumerate}
\item If addition and multiplication are defined by
  \begin{equation*}
    (\alpha,\beta) + (\gamma,\delta) = (\alpha+\gamma,\beta+\delta)
  \end{equation*}
  and
  \begin{equation*}
    (\alpha,\beta)(\gamma,\delta) = (\alpha\gamma,\beta\delta),
  \end{equation*}
  does $\F$ become a field?
  \begin{solution}
    If $\F$ were to be a field with the above operations, then the
    additive identity would have to be $(0, 0)$ and the multiplicative
    identity would be $(1,1)$. But then, for example, the element
    $(0, 1)$ would have no multiplicative inverse since for all real
    $a$ and $b$, $(a,b)(0,1) = (0,b) \neq (1,1)$. It follows that $\F$
    is not a field.
  \end{solution}
\item If addition and multiplication are defined by
  \begin{equation*}
    (\alpha, \beta) + (\gamma, \delta) = (\alpha + \gamma, \beta + \delta)
  \end{equation*}
  and
  \begin{equation*}
    (\alpha, \beta)(\gamma,\delta) = (\alpha\gamma - \beta\delta,
    \alpha\delta + \beta\gamma),
  \end{equation*}
  is $\F$ a field then?
  \begin{solution}
    Yes, $\F$ is a field in this case. In fact, $\F$ is isomorphic to
    the complex numbers $\C$ (this is actually one way of defining the
    complex numbers). Here $(0,0)$ takes the role of the additive
    identity, $(1,0)$ takes the role of the multiplicative identity,
    and any complex number $a + bi$ corresponds to the element $(a,b)$
    in $\F$.
  \end{solution}
\item What happens (in both the preceding cases) if we consider
  ordered pairs of complex numbers instead?
  \begin{solution}
    In the first case $\F$ is not a field for the same reason given
    above. The second case is more interesting, but it is not a field
    either. Consider,
    \begin{equation*}
      (1, i)(1, -i) = (1 - 1, 0) = (0, 0).
    \end{equation*}
    In this case we see that $\F$ has two nonzero elements whose
    product is zero, but this is not possible for a field, as was
    proven in Exercise~\ref{exercise:field-properties}.
  \end{solution}
\end{enumerate}
