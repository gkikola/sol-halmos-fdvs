\chapter{Spaces}

\section{Fields}

\Exercise1 Almost all the laws of elementary arithmetic are
consequences of the axioms defining a field. Prove, in particular,
that if $\F$ is a field, and if $\alpha$, $\beta$, and $\gamma$ belong
to $\F$, then the following relations hold.
\begin{enumerate}
\item $0 + \alpha = \alpha$.
  \begin{proof}
    By the commutativity of addition and the definition of $0$,
    \begin{equation*}
      0 + \alpha = \alpha + 0 = \alpha. \qedhere
    \end{equation*}
  \end{proof}
\item If $\alpha + \beta = \alpha + \gamma$, then $\beta = \gamma$.
  \begin{proof}
    Adding $-\alpha$ to both sides of the first equation gives
    \begin{equation*}
      (\alpha + \beta) + (-\alpha) = (\alpha + \gamma) + (-\alpha),
    \end{equation*}
    which by associativity and commutativity of addition gives
    \begin{equation*}
      (\alpha + (-\alpha)) + \beta = (\alpha + (-\alpha)) + \gamma,
    \end{equation*}
    and by definition of additive inverses, this gives
    \begin{equation*}
      0 + \beta = 0 + \gamma
      \quad\text{or}\quad
      \beta = \gamma,
    \end{equation*}
    making use of the fact that $0 + \delta = \delta$ for any
    $\delta\in\F$ (already proven above).
  \end{proof}
\item $\alpha + (\beta - \alpha) = \beta$. (Here
  $\beta - \alpha = \beta + (-\alpha)$.)
  \begin{proof}
    This follows from commutativity and associativity of addition:
    \begin{equation*}
      \alpha + (\beta - \alpha)
      = \alpha + (-\alpha + \beta)
      = (\alpha - \alpha) + \beta
      = 0 + \beta
      = \beta. \qedhere
    \end{equation*}
  \end{proof}
\item $\alpha\cdot0 = 0\cdot\alpha = 0$.
  \begin{proof}
    By definition of $0$ and $1$ and by distributivity we have
    \begin{equation*}
      \alpha\cdot0 = \alpha(1 - 1)
      = \alpha\cdot1 - \alpha\cdot1
      = \alpha - \alpha = 0.
    \end{equation*}
    By commutativity of multiplication, $0\cdot\alpha = 0$ as well.
  \end{proof}
\item $(-1)\alpha = -\alpha$.
  \begin{proof}
    From the various field axioms we have
    \begin{align*}
      (-1)\alpha
      &= 0 + (-1)\alpha
        = (\alpha - \alpha) + (-1)\alpha \\
      &= (-\alpha + \alpha) + \alpha(-1) \\
      &= -\alpha + (\alpha\cdot1 + \alpha(-1)) \\
      &= -\alpha + \alpha(1 - 1) \\
      &= -\alpha + \alpha\cdot0 \\
      &= -\alpha + 0
        = -\alpha. \qedhere
    \end{align*}
  \end{proof}
\item $(-\alpha)(-\beta) = \alpha\beta$.
  \begin{proof}
    Since
    \begin{align*}
      (-1)(-1) + (-1) &= (-1)(-1) + (-1)1 \\
                      &= (-1)(-1 + 1) = -1\cdot0 = 0,
    \end{align*}
    it follows that $(-1)(-1)$ is an additive inverse of $-1$. Since
    additive inverses are unique, we have $(-1)(-1) = 1$. Using this
    fact along with the previous result and with commutativity and
    associativity of multiplication we have
    \begin{equation*}
      (-\alpha)(-\beta)
      = ((-1)\alpha)((-1)\beta)
      = ((-1)(-1))(\alpha\beta)
      = 1(\alpha\beta) = \alpha\beta
    \end{equation*}
    as desired.
  \end{proof}
\item If $\alpha\beta = 0$, then either $\alpha = 0$ or $\beta = 0$
  (or both).
  \begin{proof}
    Let $\alpha\beta = 0$. If $\alpha = 0$ then we are done, so
    suppose $\alpha$ is nonzero. Then $\alpha$ has a unique
    multiplicative inverse $\alpha^{-1}$. Multiplying both sides of
    the original equation by this inverse gives
    \begin{equation*}
      \alpha^{-1}(\alpha\beta) = \alpha^{-1}\cdot0
    \end{equation*}
    which gives
    \begin{equation*}
      (\alpha^{-1}\alpha)\beta = 0.
    \end{equation*}
    And since $\alpha^{-1}\alpha = \alpha\alpha^{-1} = 1$ we have
    $\beta = 0$ which completes the proof.
  \end{proof}
\end{enumerate}
