\sectionnum{4}
\section{Vector Spaces, Examples, and Comments}

\Exercise1 Prove that if $x$ and $y$ are vectors and if $\alpha$ is a
scalar, then the following relations hold.
\label{exercise:spaces:vspace-properties}
\begin{enumerate}
\item $0 + x = x$.
  \begin{proof}
    This follows directly from the commutativity of addition of
    vectors along with the definition of the vector $0$.
  \end{proof}
\item $-0 = 0$.
  \begin{proof}
    By definition of $0$, we have $-0 + 0 = -0$. But the left-hand
    side is equal to $0$ by definition of additive inverses, so
    $0 = -0$.
  \end{proof}
\item $\alpha\cdot0 = 0$.
  \begin{proof}
    Since $0 = 0 + 0$, we have by distributivity that
    \begin{equation*}
      \alpha\cdot0 = \alpha(0 + 0) = \alpha\cdot0 + \alpha\cdot0.
    \end{equation*}
    Adding $-(\alpha\cdot0)$ to both sides then gives
    $0 = \alpha\cdot0$ as desired.
  \end{proof}
\item $0\cdot x = 0$.
  \begin{proof}
    The proof is similar to the previous one, except that the $0$ on
    the left is a scalar:
    \begin{equation*}
      0x = (0 + 0)x = 0x + 0x,
    \end{equation*}
    and adding $-(0x)$ to both sides gives the desired result.
  \end{proof}
\item If $\alpha x = 0$, then either $\alpha = 0$ or $x = 0$ (or
  both).
  \begin{proof}
    Let $\alpha x = 0$. If $\alpha = 0$ then we are done, so suppose
    $\alpha$ is nonzero. Then $\alpha\in\F$ has an inverse
    $\alpha^{-1}$. Multiplying both sides of $\alpha x = 0$ by
    $\alpha^{-1}$ then gives
    \begin{equation*}
      \alpha^{-1}(\alpha x) = \alpha^{-1}\cdot0,
    \end{equation*}
    which implies
    \begin{equation*}
      1x = x = 0,
    \end{equation*}
    completing the proof.
  \end{proof}
\item $-x = (-1)x$.
  \begin{proof}
    By distributivity we have
    \begin{equation*}
      x + (-1)x = 1x + (-1)x = (1 + -1)x = 0x = 0.
    \end{equation*}
    Now adding $-x$ to both sides and simplifying gives $(-1)x = -x$
    as required.
  \end{proof}
\item $y + (x - y) = x$. (Here $x - y = x + (-y)$.)
  \begin{proof}
    Using commutativity and associativity of addition gives
    \begin{equation*}
      y + (x - y) = (y + (-y)) + x = 0 + x = x.\qedhere
    \end{equation*}
  \end{proof}
\end{enumerate}

\Exercise2 If $p$ is a prime, then $\Z_p^n$ is a vector space over
$\Z_p$; how many vectors are there in this vector space?
\begin{solution}
  Since $\Z_p$ has $p$ members, the number of possible vectors is
  $p^n$.
\end{solution}

\Exercise3 Let $\V$ be the set of all (ordered) pairs of real
numbers. If $x = (\xi_1,\xi_2)$ and $y = (\eta_1,\eta_2)$ are elements
of $\V$, write
\begin{align*}
  x + y &= (\xi_1 + \eta_1, \xi_2 + \eta_2) \\
  \alpha x &= (\alpha\xi_1, 0) \\
  0 &= (0, 0) \\
  -x &= (-\xi_1, -\xi_2).
\end{align*}
Is $\V$ a vector space with respect to these definitions of the linear
operations? Why?
\begin{solution}
  $\V$ is not a vector space since $1x \neq x$ when $\xi_2$ is
  nonzero.
\end{solution}
