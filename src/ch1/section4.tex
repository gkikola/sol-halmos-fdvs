\sectionnum{4}
\section{Vector Spaces, Examples, and Comments}

\Exercise1 Prove that if $x$ and $y$ are vectors and if $\alpha$ is a
scalar, then the following relations hold.
\label{exercise:spaces:vspace-properties}
\begin{enumerate}
\item $0 + x = x$.
  \begin{proof}
    This follows directly from the commutativity of addition of
    vectors along with the definition of the vector $0$.
  \end{proof}
\item $-0 = 0$.
  \begin{proof}
    By definition of $0$, we have $-0 + 0 = -0$. But the left-hand
    side is equal to $0$ by definition of additive inverses, so
    $0 = -0$.
  \end{proof}
\item $\alpha\cdot0 = 0$.
  \begin{proof}
    Since $0 = 0 + 0$, we have by distributivity that
    \begin{equation*}
      \alpha\cdot0 = \alpha(0 + 0) = \alpha\cdot0 + \alpha\cdot0.
    \end{equation*}
    Adding $-(\alpha\cdot0)$ to both sides then gives
    $0 = \alpha\cdot0$ as desired.
  \end{proof}
\item $0\cdot x = 0$.
  \begin{proof}
    The proof is similar to the previous one, except that the $0$ on
    the left is a scalar:
    \begin{equation*}
      0x = (0 + 0)x = 0x + 0x,
    \end{equation*}
    and adding $-(0x)$ to both sides gives the desired result.
  \end{proof}
\item If $\alpha x = 0$, then either $\alpha = 0$ or $x = 0$ (or
  both).
  \begin{proof}
    Let $\alpha x = 0$. If $\alpha = 0$ then we are done, so suppose
    $\alpha$ is nonzero. Then $\alpha\in\F$ has an inverse
    $\alpha^{-1}$. Multiplying both sides of $\alpha x = 0$ by
    $\alpha^{-1}$ then gives
    \begin{equation*}
      \alpha^{-1}(\alpha x) = \alpha^{-1}\cdot0,
    \end{equation*}
    which implies
    \begin{equation*}
      1x = x = 0,
    \end{equation*}
    completing the proof.
  \end{proof}
\item $-x = (-1)x$.
  \begin{proof}
    By distributivity we have
    \begin{equation*}
      x + (-1)x = 1x + (-1)x = (1 + -1)x = 0x = 0.
    \end{equation*}
    Now adding $-x$ to both sides and simplifying gives $(-1)x = -x$
    as required.
  \end{proof}
\item $y + (x - y) = x$. (Here $x - y = x + (-y)$.)
  \begin{proof}
    Using commutativity and associativity of addition gives
    \begin{equation*}
      y + (x - y) = (y + (-y)) + x = 0 + x = x.\qedhere
    \end{equation*}
  \end{proof}
\end{enumerate}

\Exercise2 If $p$ is a prime, then $\Z_p^n$ is a vector space over
$\Z_p$; how many vectors are there in this vector space?
\begin{solution}
  Since $\Z_p$ has $p$ members, the number of possible vectors is
  $p^n$.
\end{solution}

\Exercise3 Let $\V$ be the set of all (ordered) pairs of real
numbers. If $x = (\xi_1,\xi_2)$ and $y = (\eta_1,\eta_2)$ are elements
of $\V$, write
\begin{align*}
  x + y &= (\xi_1 + \eta_1, \xi_2 + \eta_2) \\
  \alpha x &= (\alpha\xi_1, 0) \\
  0 &= (0, 0) \\
  -x &= (-\xi_1, -\xi_2).
\end{align*}
Is $\V$ a vector space with respect to these definitions of the linear
operations? Why?
\begin{solution}
  $\V$ is not a vector space since $1x \neq x$ when $\xi_2$ is
  nonzero.
\end{solution}

\Exercise4 Sometimes a subset of a vector space is itself a vector
space (with respect to the linear operations already given). Consider,
for example, the vector space $\C^3$ and the subsets $\V$ of $\C^3$
consisting of those vectors $(\xi_1, \xi_2, \xi_3)$ for which
\begin{enumerate}
\item $\xi_1$ is real,
\item $\xi_1 = 0$,
\item either $\xi_1 = 0$ or $\xi_2 = 0$,
\item $\xi_1 + \xi_2 = 0$,
\item $\xi_1 + \xi_2 = 1$.
\end{enumerate}
In which of these cases is $\V$ a vector space?
\begin{solution}
  To check whether a subset $\V$ of a vector space is a subspace, we
  only need to ensure three things: that the subset $\V$ is nonempty,
  that for all $x,y\in\V$ also $x + y\in\V$, and that for all
  $\alpha\in\F$ and $x\in\V$ also $\alpha x\in\V$. It is then easy to
  check that all of the axioms hold. For example, since $\V$ must be
  nonempty, we can take $x\in\V$ so that $0x = 0\in\V$. The other
  properties are similar.
  \begin{enumerate}
  \item This is a vector space over $\R$ since it is nonempty, it is
    closed under vector addition, and it is closed under scalar
    multiplication.

    However, this is not a vector space over the field $\C$ since
    scalar multiplication is not closed. For example, $(1,1,1)\in\V$
    but $i(1, 1, 1)\not\in\V$.
  \item $0\in\V$ so $\V$ is nonempty. For any $x = (0,\xi_2,\xi_3)$
    and $y = (0,\eta_2,\eta_3)\in\V$ we have $x + y\in\V$ so $\V$ is
    closed under addition. And for any $\alpha\in\C$ we have
    $\alpha x = (0, \alpha\xi_2, \alpha\xi_3)\in\V$ so $\V$ is closed
    under scalar multiplication. Therefore $\V$ is a vector space.
  \item Since $(1, 0, 0)\in\V$ and $(0, 1, 0)\in\V$ but their sum
    $(1, 1, 0)\not\in\V$, we see that $\V$ is not a vector space.
  \item This $\V$ is obviously not closed under scalar multiplication,
    so it cannot be a vector space.
  \item This $\V$ does not contain the zero vector, so it cannot be a
    vector space. \qedhere
  \end{enumerate}
\end{solution}

\Exercise5 Consider the vector space $\P$ and the subsets $\V$ of $\P$
consisting of those vectors (polynomials) $x$ for which
\begin{enumerate}
\item $x$ has degree $3$,
\item $2x(0) = x(1)$,
\item $x(t) \geqq 0$ whenever $0\leqq t\leqq1$,
\item $x(t) = x(1 - t)$ for all $t$.
\end{enumerate}
In which of these cases is $\V$ a vector space?
\begin{solution}
  \begin{enumerate}
  \item This $\V$ is not closed under addition. For example, the
    polynomials $t^3$ and $-t^3 + t^2$ are both in $\V$ but their sum
    is not (it has degree $2$).
  \item $\V$ is nonempty since $0\in\V$. Suppose $x,y\in\V$. Then
    \begin{equation*}
      2(x + y)(0) = 2(x(0) + y(0))
      = 2x(0) + 2y(0) = x(1) + y(1) = (x + y)(1)
    \end{equation*}
    and for any $\alpha\in\C$,
    \begin{equation*}
      2(\alpha x)(0) = 2\alpha(x(0)) = \alpha(2x(0))
      = \alpha(x(1)) = (\alpha x)(1).
    \end{equation*}
    Therefore $\V$ is a vector space.
  \item This $\V$ is not closed under scalar multiplication. For
    example, $t\in\V$ but $(-1)t = -t\not\in\V$.
  \item $0\in\V$ so $\V$ is nonempty. And for any $x,y\in\V$ and
    $\alpha\in\C$, we have
    \begin{equation*}
      (x + y)(t) = x(t) + y(t) = x(1 - t) + y(1 - t) = (x + y)(1 - t)
    \end{equation*}
    and
    \begin{equation*}
      (\alpha x)(t) = \alpha(x(t)) = \alpha(x(1 - t)) = (\alpha x)(1 - t),
    \end{equation*}
    so $\V$ is a vector space. \qedhere
  \end{enumerate}
\end{solution}
