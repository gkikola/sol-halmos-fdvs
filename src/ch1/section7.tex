\sectionnum{7}
\section{Linear Dependence, Linear Combinations, and Bases}

\Exercise1
\begin{enumerate}
\item Prove that the four vectors
  \begin{align*}
    x &= (1, 0, 0), \\
    y &= (0, 1, 0), \\
    z &= (0, 0, 1), \\
    u &= (1, 1, 1),
  \end{align*}
  in $\C^3$ form a linearly dependent set, but any three of them are
  linearly independent.
  \begin{proof}
    The vectors $x$, $y$, $z$, and $u$ are linearly dependent since
    \begin{equation*}
      x + y + z - u = 0.
    \end{equation*}
    However any three them form a linearly independent set, as we will
    now show. It should be clear that the set $\{x,y,z\}$ is linearly
    independent, so we just need to show that any two of $x,y,z$ taken
    together with $u$ forms a linearly independent set.

    There are three cases, but we can handle them all at once by
    observing the symmetry in the coordinates of $x$, $y$, and
    $z$. If, for example, $x$, $y$, and $u$ are independent, then
    exactly the same reasoning and indeed the same system of equations
    (with some variable indices switched around) will show that $x$,
    $z$, and $u$ are independent and that $y$, $z$, and $u$ are
    also. Therefore we will only show the linear independence of $x$,
    $y$, and $u$, and that will suffice.

    So let
    \begin{equation*}
      \alpha_1x + \alpha_2y + \alpha_3u = 0.
    \end{equation*}
    By equating the third coordinate of each side we get
    $\alpha_3 = 0$. Then from the first coordinates we get
    $\alpha_1 + \alpha_3 = 0$ so that $\alpha_1 = 0$. Similarly, from
    the second coordinates we get $\alpha_2 = 0$. Therefore
    $\alpha_1,\alpha_2,\alpha_3$ must all be zero, which establishes
    the linear independence of the three vectors, completing the
    proof.
  \end{proof}
\item If the vectors $x$, $y$, $z$, and $u$ in $\P$ are defined by
  $x(t) = 1$, $y(t) = t$, $z(t) = t^2$, and $u(t) = 1 + t + t^2$,
  prove that $x$, $y$, $z$, and $u$ are linearly dependent, but any
  three of them are linearly independent.
  \begin{proof}
    Observe that there is a natural correspondence between these
    vectors and the vectors in the previous part of this exercise:
    each of $x$, $y$, and $z$ have one coefficient equal to $1$ and
    the others $0$, while $u$ has all coefficients equal to $1$. So
    the same relation
    \begin{equation*}
      x + y + z - u = 0
    \end{equation*}
    shows that the four vectors are linearly dependent, while exactly
    the same reasoning as before will show that any three of them are
    linearly independent.
  \end{proof}
\end{enumerate}
